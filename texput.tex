%%% Local Variables:
%%% mode: latex
%%% TeX-master: "<none>"
%%% End:

\title{Lempel and Ziv 1977 (LZ77)}

\author{Vicente González Ruiz}

\maketitle

\section{History}

\begin{itemize}
\tightlist
\item
  Jacov Ziv and Abraham Lempel proposed the LZ77 algorithm in 1977~\cite{ziv1977universal}.
\item
  In the eighties,
  \href{https://en.wikipedia.org/wiki/Lempel\%E2\%80\%93Ziv\%E2\%80\%93Storer\%E2\%80\%93Szymanski}{LZSS}
  (a branch of LZ77) was implemented by Haruyasu Yoshizaki~\cite{storer1982data} (and implemented in the
  \href{https://en.wikipedia.org/wiki/LHA_(file_format)}{LHA
  compressor}, discovering the possibilities of the LZ77 encoder.
\item
  After that, a large number of text compressors have been based on LZ77
  (or a variation of it). Some of the most famous are:
  \href{https://en.wikipedia.org/wiki/ARJ}{ARJ},
  {[}RAR{]}(https://en.wikipedia.org/wiki/RAR\_(file\_format),
  \href{https://en.wikipedia.org/wiki/Gzip}{gzip} and
  \href{https://en.wikipedia.org/wiki/7z}{7z}.
\item
  LZ77 processes a sequence of symbols using the structure:
\end{itemize}

\imgw{600}{graphics/LZ77.png}

\begin{itemize}
\tightlist
\item
  The dictionary and the look-ahead buffer have a fixed size and can be
  considered as a sliding window moving over the symbols while they are
  coded. In each iteration, the input of new symbols to the buffer
  generates the output of the oldest ones, which become the newest
  symbols of the dictionary.
\end{itemize}

\section{Encoder}

\begin{enumerate}
\tightlist
\item
  Let \(I\) the length of the dictionary and \(J\) the length of the
  buffer.
\item
  Input the first \(J\) symbols in the buffer.
\item
  While the input is not exhausted:

  \begin{enumerate}
  \tightlist
  \item
    Let \(i\) the position of the first \(j\) symbols of the buffer in
    the dictionary, and \(k\) the symbol that makes that \(j\) can not
    be larger.
  \item
    Output \(ijk\).
  \item
    Input the next \(j+1\) symbols in the buffer.
  \end{enumerate}
\end{enumerate}

\section{Example}
\imgw{800}{graphics/LZ77_encoding_example.png}

\section{Decoder}

\begin{enumerate}
\tightlist
\item
  While code-words \(ijk\) are input:

  \begin{enumerate}
  \tightlist
  \item
    Output the \(j\) symbols extracted from the position \(i\) in the
    dictionary.
  \item
    Output \(k\).
  \item
    Put all the decoded symbols in the beginnig of the buffer.
  \end{enumerate}
\end{enumerate}

\subsection{Example}
\imgw{500}{graphics/LZ77_decoding_example.png}
\begin{itemize}
\tightlist
\item
  Parameters \(I\) and \(J\) control the performance of the algorithm.
  They should be large enough to guarantee the matching of long strings,
  but should be small in order to reduce the number of bits of the
  code-words \(ijk\). Typical sizes are: \(\log_2(I)=12.0\) and
  \(\log_2(J)=4.0\).
\end{itemize}

\section{Lab: LZSS}
\href{https://nbviewer.jupyter.org/github/vicente-gonzalez-ruiz/LZ77/blob/master/LZSS.ipynb}{IPython notebook}

\bibliography{text-compression}
